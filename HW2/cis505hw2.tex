\documentclass[english]{article}

\usepackage[latin9]{inputenc}
\usepackage[letterpaper]{geometry}
\geometry{verbose,tmargin=1in,bmargin=1in,lmargin=1in,rmargin=1in}
\usepackage{graphicx}
\usepackage{amsmath}
\usepackage{amssymb}
\usepackage{listings}
\usepackage{tikz}
\usetikzlibrary{arrows}
%\usetikzlibrary{graphdrawing}
%\usetikzlibrary{graphs}
%\usegdlibrary{trees}

\title{ESE 605 Homework 3}
\author{Stephen Phillips}
\date{\today }


\begin{document}
\maketitle
\subsection*{Problem 1}
\subsubsection*{Part a}
Due to network conditions, the transmission time between one machine and 
another can vary quite a bit. Therefore by taking statistics of previous
transmission times, we can lower the error of the clock skew measurement
by getting a better measurement of the actual transmission time. We do this by
sending multiple messages then taking an average or the median to determine
the acutal transmission time. Then 


\subsubsection*{Part b}
It doesn't make a difference, since the results of both will give the process
with the highest process ID the position of leadership. So the outcome of
both elections will be the same and the final outcome will not change. There
may be more wasted communication, but it won't affect how the system behaves.


\subsection*{Problem 2}
\subsubsection*{Part a}
Reading and Writing files from a file server are best done with
\textbf{at least once} semantics. Reading a file clearly is idempotnent, and 
since we specify the exact positions in the file to write to, and the data
to write to it, then the write becomes idempotent. Since these are both
idempotnent operations, to guarantee their success, we can send the message
as many times as we want to.

\subsubsection*{Part b}
Compiling a program should be done with \textbf{at least once} semantics since
the compiled code only depends on the source code files. Because of this,
commands to compile the code can be sent more than once. While this may lead
to unnecessary computation, it will guarantee that the code is compiled which
is what we want more.


\subsubsection*{Part c}
Online banking should be \textbf{at most once} since the transactions need to
happen only once. Ideally we would use exactly once, but unfortunately, that
cannot be implemented in the real world. So as we do not want to for example
send money to an account twice, we use at most once semantics. We could make
this to at least once if we were clever, using transaction IDs or something
but barring complicated implementation you need to use at most once.

\subsection*{Problem 3}
\subsubsection*{Part a}
Lamport's logical clock gives the guarantee if $A \rightarrow B$
($B$ happens after knowledge of $A$)
then $L(A) < L(B)$ (logical clock value of $A$ less than the
logical clock value of $B$). This is enforced by how Lamport's logical clock
is run. Since every event within a process increments the logical clock,
within a process this holds. Between processes, since when a process gets a
message it sets is clock to the max of its own clock and the message clock
plus 1. Therefore it maintains this invariant. Concurrent events are not
considered.

\subsubsection*{Part b}
Since each process runs on its own logical clock then the reverse of part a,
that $L(A) < L(B)$ means $A \rightarrow B$, is not true. For instance consider
3 processes $P_1$, $P_2$, and $P_3$. Let's say $P_1$ sends a message to $P_2$
and $P_3$ with time stamp 9. Then $P_2$ sends a message with time stamp 6 to
the other processes. From these time stamps we can say nothing about which
came first, whether one came before the other or they happened concurrently.
So even though $P_2$ sent its message later it has the smaller timestamp.

Vector clocks fix this by maintaining logical clocks for each of the
processes in the system. So each process knows the logical clock for
all the other processes, and uses them to determine late messages. For instance
say we have processes $P_1$, $P_2$, and $P_3$ again. This time they are 
maintaining vector clocks. Say that $P_1$ sends a message to the others, 
with time stamp say (2,0,0) and then $P_2$ gets it and sends another message
with time stamp say (2,3,0). If $P_3$ gets the message from $P_2$ first,
with time stamp (2,3,0), it knows it has missed a message from $P_1$
since it has a clock of something like (0,0,1). And therefore waits until
it has recieved $P_1$'s message with time stamp (2,0,0) before it sends
any of its messages to the application that it is running.











\end{document}
